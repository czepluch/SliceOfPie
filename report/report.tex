\documentclass[a4paper,11pt]{article}
\usepackage[T1]{fontenc}
\usepackage{inputenc}
\usepackage{amsfonts}
\usepackage{graphicx}
\usepackage{bm}
\usepackage{varioref}
\usepackage[english]{babel}
\usepackage{hyperref}
\usepackage{tikz}
\usetikzlibrary{arrows,decorations.pathmorphing,backgrounds,positioning,fit,petri}
\usepackage{enumitem}
\usepackage{newclude}
\newcommand{\field} [1] {\mathbb{#1}}
\begin{document}

\begin{titlepage}
\centering \parindent=0pt
\newcommand{\HRule}{\rule{\textwidth}{1mm}}
\vspace*{\stretch{1}} \HRule\\[1cm]\Huge\bfseries
Semester Project: "Slice of Pie"\\[0.7cm]
\HRule\\[4cm]  \large Group 8:
\\Jacob Stenum Czepluch (jstc@itu.dk), \\Niels Roesen Abildgaard (nroe@itu.dk), \\Niclas Tollstorff (nben@itu.dk), \\Sigurt Bladt Dinesen (sidi@itu.dk), \\Michael Storgaard (mmun@itu.dk)\\
\vspace*{\stretch{2}} \normalsize %
\begin{flushleft}
Semester Project\\
Bachelor in Software Development\\
IT-University of Copenhagen\\
Jacob Bardram (bardram@itu.dk)\\
Dario Pacino(dpacino@itu.dk) \\
December 17, 2012 \end{flushleft}
\end{titlepage}

\tableofcontents
\pagebreak

% This is how a section is created.
\pagebreak
\section{Introduction}
\include*{Introduction/introduction}

\pagebreak
\section{User manual}
\include*{User_manual/main}

% This is how a subsection is created.
\pagebreak
\subsection{Noob manual}
\label{sec:User manual}
\include*{User_manual/noob}

\end{document}
