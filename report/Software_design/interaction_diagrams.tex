In addition to the class diagrams we picked out two of our use cases that we thought to be critical for the system. For each of these we made an interaction diagram
to make sure that these parts of the application are implemented correctly. 

The purposes of our interaction diagrams are to visualize the interactive behaviour of our system. Interaction diagrams help us capture the dynamic behaviour of 
the system. It also describes the message flow in the system, the structural organization of the objects, and the interaction among objects.

\subsubsection{Interaction Diagram for Use Case 2}

We have chosen use case 2 - 'Share Project' to be one of the use cases that we make an interaction diagram for since sharing a project is one of the essential
requirements of this project. 

We want the user to be able to invite more than one user at the same time. This is done through a list of comma separated email addresses. 

%Insert illustration here

As the interaction diagram shows, the user chooses a project to share, and give a list of mail addresses of the other users to be invited. This can be done through
either one of the UIs. We have however in this example user the WebUI. 

Each mail address in the list is then invited to the project. Since this only is a proof of concept, we have not put any effort into making sure that invites 
are sent to the users being invited. Instead, when a project is being shared with another user, the project will just show up in the user's project overview.

If a user that does not exist is invited with an email not already existing in the databse, the project will still be shared with that user. 
This means that if a user with the specified email then registers with this email address, the project will already be shared with him. 

This is of course not how we would have done it in a real world application but since this is only a proof of concept, we think this suffices. 
