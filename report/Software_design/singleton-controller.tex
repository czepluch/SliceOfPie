\subsubsection{Singleton pattern in the Controller}

The Singleton pattern limits the number of existing versions of a class to one. In SliceOfPie,
the Controller (which is the entry point for the general API of the SliceOfPie library; see
Section \ref{sec:implementationview}) is a Singleton, meaning that any running instance of
an application can only have one instance of the controller existing.

The singleton pattern is usually useful when a state is necessary, but only one instance of the
class should exist. If no state is required, it might as well be a static class with static
methods.

In the case of the controller, the state is which implementation of \emph{IFileModel} is used.
This is decided by the value of the controller's property \emph{IsWebController}. If this boolean
is set to \verb|true|, the controller will be using the WebFileModel internally. Of course, a
client using the API should not have to think about this, and should merely make the distinction
of whether files should be saved on the local filesystem or not.