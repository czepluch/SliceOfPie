\subsection{Process view}

In \SOP{}, concurrency is handled in the controller.

To handle concurrent calls to the controller and to keep user interfaces from blocking when methods
are being called, we have implemented a pattern called the Asynchronous Programming Model (APM; see
Section \ref{sec:APM}).

APM allows for asynchronous calls by allowing a callback to fire when a call finishes. This means that
a client can change from using a method to using its counterparts prefixed with \emph{Begin} and
\emph{End} to opt in for an asynchronous call. This is particularly useful in desktop clients, which
would usually lock up (become unresponsive) while waiting for the longer blocking calls to conclude.

\begin{figure}[htb]
	\centering
	\includegraphics[width=1\textwidth]{Software_architecture/graphics/apm-sequence.png}
	\caption{Sequence Diagram showing the inner workings of any APM call. Instead of using a new Thread, our implementation
        delegates calls to the Thread Pool, to optimize performance (reducing overhead of creating new threads).}
	\label{fig:apm-sequence}
\end{figure}