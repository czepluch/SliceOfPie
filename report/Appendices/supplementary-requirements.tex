\section{Supplementary requirements}
\label{sec:Supplementary Requirements}

Following is a list of requirements for the system that cannot be defined as Use Cases %Insert reference%
but are still important to the project.

\subsection{Functionality}
The system must keep track of user projects and project files, including revision control.

The system must supply a web-interface, as well as a local interface for offline operations.

The system should provide, both locally and in the web-interface, capabilities for viewing and editing documents. Documents should be able to include images.

The system must support merging of conflicting versions of files, such as those that would occur from simultaneous offline and online editing by different users.

\subsection{Usability}
Document creation, sharing, and other technical background operations must be performed within a time span of a couple of seconds (~2).

The only operations that are allowed to take more than a couple of seconds are user operations such as editing documents and reviewing merges.

Merge reviews should be easy to understand, such that a normal page of revisions can be revised in less than 2 minutes.

\subsection{Reliability}
The system must have 98\% up-time. Planned down-time not taken into account.

The system should be able to reject errors in synchronization, such that document synchronization is atomic and correct.

\subsection{Performance}
The system is partly based on a web interface. Performance, therefore, is dependent on external factors such as the users internet connection.
This makes bandwidth consumption a prime consideration for system performance.

The local client must be light-weight enough to run on a five year old laptop with a dual-core processor.

\subsection{Supportability}
The system should be self contained enough to require technical assistance only during server setup.

The system should have high version stability. No update is allowed to break backwards-compatibility.

\subsection{Security}
Users are authenticated via a personal login and password.

Passwords may not be kept as plaintext, but are maintained as salted hashes.

\subsection{Implementation Constraints}
The system should be written in C\#. This is a requirement from a very important stakeholder.
