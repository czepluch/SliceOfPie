\subsection{Known bugs}
\label{sec:bugs}

Following is a list of bugs known in the system. These are faults that should be corrected before any public
release of the code is intended. Most of these would be quick to fix, but are accumulated here as we chose
to focus on producing a report for the last sprint (during which time we discovered most of these bugs).

\subsubsection{WPF pop-up on top of other windows}
Our current prototype makes use of the PopUp (WPF) class in the local client. The usage of the PopUp class is very well suited for fast sketching and prototyping in UIs, although despite the name the PopUp class is not originally intended to be shown as a modal dialog window. The implications of this issue is that any pop-up window in our local client will always be the topmost window (even over a Windows task manager), since that behavior is build into the class. Our reasoning behind using this class is the efficiency in prototyping, which we think has a certain value in an agile development environment. The two most obvious solutions to the "bug" would be to either use a custom modification of the Window class or use C\#'s P/Invoke feature to use the Win32 API directly and remove the behavior of the window through that (open source solutions exist for this). We think that both of these solutions are out-of-scope and a unimportant with regards to the purpose of this project (making a prototype).

\subsubsection{Exception in web client when sharing projects}
In web client exceptions thrown for failure situations, when sharing projects with non-existing email addresses, are not handled by the UI, which will cause the system to show an error page.

\subsubsection{Multiple default projects in local client}
In local client multiple default projects can be created, when using the local client on a new computer from an existing user. This is caused due to the LocalFileModel class creating the default project before running the synchronization. The program does not know exactly what directory in the file system is actually the default project, so it will just treat it as any other project and add it to the users account even though several projects already exist.

\subsubsection{Removing entities in local client}
In local client you can remove projects, folders and documents, but these deletion does not get manifested in the database, because the local client does not keep an index of entities, which results in this bug. This is because the program can not differentiate between locally deleted entities and remotely added entities, and it therefore chooses to add locally instead of always removing remotely, which by far is the better solution, but not the most optimal.

\subsubsection{Multiple Instances of Singleton Controller}
It is currently possible to get multiple Controller-objects by first getting an instance (through the \verb|Instance| property),
then changing the value of the \verb|IsWebController|-property and getting an instance again. This would result in references to
two different controllers.

IsWebController is intended to be set before instantiation of the controller, and as such should throw an exception if a client tries
to change it, and the controller has already been instantiated.

\subsubsection{Different users in local client}
If a user in the local client synchronizes, all of that persons files will be stored locally. If another user then synchronizes on the same computer, all of the first users files will be added to the second user. This could be solved by adding different local user repositories or simply adding functionality to clear the local folders before switching to a new user.