\subsection{Lack of security in password handling}

It was one of our initial non-functional requirements (Appendix \ref{sec:Supplementary Requirements}) that
users' passwords should not be kept in plain text.

How would such a requirement arise? With a recent surge in database break-ins\cite{lulzsec0711}, it has
become public knowledge that keeping passwords in plain text is insecure.

It is, however, not much more secure to simply hash passwords: in a big enough database, passwords can be
compared, and there will be a great chance that the most used password in the database corresponds with one
of the most used passwords on the internet\cite{toppasswords}.

To scramble the password in the database enough to disallow those most simple attacks (brute-force can never
be ruled out entirely), passwords should be hashed with a salt unique to that password (like the e-mail address
of the user).

In the case of SliceOfPie, the passwords are currently kept entirely in plaintext. This is a security liability,
and should not be allowed in released code, which is also what the original requirement stems from. In this first
version, a proof-of-concept, it wasn't prioritized in the top.