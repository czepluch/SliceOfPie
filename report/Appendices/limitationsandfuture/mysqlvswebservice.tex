\subsection{Security Concerns Regarding the use of Remotely Accessible MySQL Databases}
\label{sec:mysqlvswebservice}

SliceOfPie is based on a single data-server keeping the global state of the application (see \ref{sec:deployment}).

In the proof-of-concept (current) implementation of the application, the data-server is nothing more than a Linux-server
hosting a remotely accessible MySQL database. This means that any computer with access to the internet and knowledge of
the remote server's IP can try to connect to the database. Any such knowledge can be used to brute-force (or otherwise
crack open) the database. This poses a security threat.

With decompilation tools, it is possible to extract the exact IP (and in the current state also the username and password)
used to connect to the database. This means that the global state, that the entire application relies on, is - in effect -
not secure. Tampering of data is not prevented in any way.

In a production-environment (read: not-just-proof-of-concept) this would be unacceptable. An alternative exists, however:
Web-Services.

A WS hosted on the data-server could provide a publicly availible API for getting and setting data in a more controlled manner,
without leaving any database passwords hidden in the source code of the application. Providing such an API in a well-known format
such as XML or JSON\cite{ibmREST} would, additionally, allow for easy development of other types of clients for the service (browser-hosted
Javascript clients; PHP clients; etc.)

Changing the back-end from a database to a web sevice would only require changes in the file models currently communicating
with the database