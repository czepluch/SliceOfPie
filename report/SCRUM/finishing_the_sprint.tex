\subsubsection{Finishing the Sprint}

\paragraph{Product Backlog Refinement}
Before finishing the Sprint we refined the Product Backlog according to any new knowledge gained during the Sprint. This refinement is recommended as ``one of the lesser known, but valuable, guidelines in Scrum''\cite{ScrumPrimer2}, and for us it involved splitting items and re-prioritizing the Product Backlog. We did this in a short meeting working directly in Pivotal Tracker.

\paragraph{Sprint Review}
Concluding each Sprint we held a Sprint Review meeting for one hour. Ideally, in a real work environment, the Product Owner should choose to invite other stakeholders (customers etc.) to this meeting, since it is centered around inspecting and discussing the situation for the current product. This meeting functioned as a hands-on demonstration of current product capabilities and features. We discussed the product with the Product Owner in order to get valuable feedback and made sure everything was going as planned. We then proceeded with a Sprint Retrospective meeting.

\paragraph{Sprint Retrospective}
In this meeting, lasting up to 45 minutes, we talked about the Scrum process as well as our general work environment. Our ScrumMaster facilitated this meeting and made sure every member of the Team got a chance to give any negative as well as positive comments about the process. As such these meetings helped us make some improvements to our working environment before continuing with the next Sprint. For instance in the first of these meetings we found it necessary to meet at the university an extra day every week.

Pictures from Scrum meetings can be found in Appendix \ref{sec:scrum-meeting-images}.
