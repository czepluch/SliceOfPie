As mentioned in the above section the Product Owner prioritizes items in a list. This list is called the \textbf{Product Backlog}. These items will primarily be user-centric features, but can also be technical improvement goals or bugs to fix.

Since the Product Backlog exists and evolve throughout the lifetime of the product we found it useful to use an online tool for creating and maintaining it. Our choice fell on \textbf{Pivotal Tracker}, which is especially suited for agile project management and collaboration. We have strived to make our product backlog DEEP (\textbf{D}etailed appropiately, \textbf{E}stimated, \textbf{E}mergent, \textbf{P}rioritized) as is good practice in Scrum.
The use of Pivotal Tracker facilitates the DEEP approach by allowing the assignment of descriptions (\textit{detailed appropiately} attribute) and  estimation points (\textit{estimated} attribute), as well as supporting  ordering (\textit{prioritized} attribute ) and continous refinement (\textit{emergent} attribute).
