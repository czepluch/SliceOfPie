Using Scrum involves forming a cross-functional team consisting of three major roles: Product Owner, ScrumMaster, and Team (also called Development Team).

The \textbf{Product Owner} is interested in maximizing the return on investment (ROI) for the product. This is done by translating product features into a prioritized list, which should be continually refined. As such the Product Owner is responsible for the \textit{value} of the work being done. In our Scrum Team the role of Product Owner was assigned to Niels Roesen Abildgaard.

The \textbf{ScrumMaster}'s role is to help the Scrum Team learn and apply Scrum in order to increase the business value. It is important to note that the ScrumMaster is not the manager of the Team members, but instead functions as more of a coach to the Team regarding the correct use of Scrum. As such the ScrumMaster makes sure that the team understands and follows the Scrum principles. In our Scrum Team the role of ScrumMaster was assigned to Niclas Benjamin Tollstorff.

The \textbf{Team} builds the product in a cross-functional manner. This means that there is no specialist role. Every team member should be able to work on any part of the system. Due to our limited team size, we had our Product Owner and ScrumMaster work as a team member outside of meetings.