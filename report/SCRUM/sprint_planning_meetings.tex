\subsubsection{Sprint Planning Meetings} 
Our Sprints were prepared by two short meetings of one hour each. The first of the Sprint Planning Meetings was focused on the "what", while the second is focused on the "how".

In \textbf{Sprint Planning Part One}  we discussed our prioritizations of items in our Product Backlog. This part of the meeting was heavily influenced by the Product Owner as he was responsible for the Return of Investment. As such this part of the meeting was primarily focused on the Product Owner explaining his thoughts to the Team (and ScrumMaster) regarding what items should be chosen and also \textit{why} that is the case.

In \textbf{Sprint Planning Part Two} we discussed what items from the Product Backlog should be implemented in the current Sprint. An important aspect of this is meeting was that while the Product Owner heavily influences the priority in the Product Backlog, it was the Team that ultimately chose how many of those items to take on in the Sprint (with highest priority items chosen first). This division of work makes the each Sprint more reliable, as the Team knows how much work they can handle.

At this point we had to do some capacity planning, and while it was hard to plan the Team's capacity for the very first sprint, we could use Pivotal Tracker's running calculation of \textit{Velocity} in the later sprints, thus making the capacity planning more accurately based on the Teams performance. We've based this approach for the planning on the \textit{Yesterday's Weather} principle\cite{fowleryesterday}.

The velocity approach worked very well for us, although it does mean that we have no capacity planning artifact, as the velocity tracker was built directly into our SCRUM environment and updated continuously.

At the end of the Sprint Planning Meetings we placed the chosen items from the Product Backlog into a \textbf{Sprint Backlog} in Pivotal Tracker.