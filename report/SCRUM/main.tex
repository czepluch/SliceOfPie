\section{SCRUM}

In this project we have been using the Scrum development framework in order to develop our software solution in an iterative and incremental way. Scrum can be very briefly summarized as follows:

A cross-functional \textbf{Team} works in timeboxed periods (called \textbf{Sprints}) throughout the development. At the beginning of these Sprints the Team chooses prioritized \textbf{items} from an existing \textbf{Product Backlog} that should be implemented. These items will be added to a \textbf{Sprint Backlog} and should result in a \textbf{ Potentially Shippable Product Increment} by the end of the sprint.

After the Sprint is initiated from two short \textbf{Sprint Planning Meetings} progress is tracked with \textbf{Daily Scrum}-meetings together with \textbf{Burndown Charts} throughout the Sprints. An important point is that Sprints are never extended. If the work on an item is not considered complete according to the Teams \textbf{Definition of Done} by the end of the sprint, it is placed back in the backlog.

\textbf{Review-} and \textbf{Retrospective Meetings} at the end of each sprint makes the Team learn from any mistakes and problems, and as such Scrum is based on feedback-cycles.

This section elaborates on our use of Scrum throughout the process. Our practice of Scrum is based on The Scrum Primer 2.0\cite{ScrumPrimer2}.

\include*{SCRUM/distribution_of_scrum_roles}

\include*{SCRUM/the_product_backlog}

\include*{SCRUM/definition_of_done}

\include*{SCRUM/the_sprint_cycle}