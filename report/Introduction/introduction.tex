This report covers our semester project for the third semester at the MSc in Software Development at the IT-University of Copenhagen. This project has been made in November and December in the year 2012. 

To make this project, we were asked to form groups of the size of five persons. 

The main purpose of this project is to make an interactive document sharing system, called Slice of Pie, that merges some of the functionality of Google Drive with the synchronizing functionality of Dropbox.
 
Like Google Drive, Slice of Pie, allows the user to create, edit, and share documents through a web interface. Documents can be arranges into projects and folders. The modifications of shared documents are merged together.

Like Dropbox, Slice of Pie, gives the user the possibility to synchronize the entire Slice of Pie library into a local folder. A stand-alone client application is provided which can perform the same operations as the web client. The difference is that now the user can also work offline.

Our advisor throughout this project has been Jacob Bardram and Dario Pacino, from whom we have received both guidance and relevant lectures. We have also received feedback from our teachers assistant, Simon Bang Terkildsen, who has been of great help to us.
